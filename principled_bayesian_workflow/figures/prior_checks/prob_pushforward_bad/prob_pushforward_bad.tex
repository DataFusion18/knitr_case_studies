\documentclass{standalone}
\usepackage{graphicx}	
\usepackage{amssymb, amsmath}
\usepackage{color}

\usepackage{tikz}
\usetikzlibrary{intersections, backgrounds, math}
\usepackage{pgfmath}

\definecolor{light}{RGB}{220, 188, 188}
\definecolor{mid}{RGB}{185, 124, 124}
\definecolor{dark}{RGB}{143, 39, 39}
\definecolor{highlight}{RGB}{180, 31, 180}
\definecolor{gray10}{gray}{0.1}
\definecolor{gray20}{gray}{0.2}
\definecolor{gray30}{gray}{0.3}
\definecolor{gray40}{gray}{0.4}
\definecolor{gray60}{gray}{0.6}
\definecolor{gray70}{gray}{0.7}
\definecolor{gray80}{gray}{0.8}
\definecolor{gray90}{gray}{0.9}
\definecolor{gray95}{gray}{0.95}

\tikzmath{
  function logit_normal(\x, \s) {
    if \x <= 0 then {
      return 0;
    } else {
      if \x >= 1 then {
        return 0;
      } else {
        return exp(-0.5 * ((ln(\x / (1 - \x))) / \s) * (ln(\x / (1 - \x))) / \s)) / sqrt(6.283185 * \s * \s) / (\x * (1 - \x));
      };
    };
  };
}

\begin{document}

\begin{tikzpicture}[scale=0.3, thick]

  \fill[color=dark, domain=-10.075:10.075, smooth, samples=150, variable=\x] 
    plot ({\x}, {1 * logit_normal({(\x + 10) / 20}, 3.1622)});

  \node[] at (-13, 5) { $\pi_{p(\Theta)} (p)$ };

  \draw [-, >=stealth, line width=1] (-10.05, 0) -- +(20.1, 0);
  \draw [->, >=stealth, line width=1] (-10, -0.05) -- +(0, 11);
  \draw [->, >=stealth, line width=1] (10, -0.05) -- +(0, 11);
  \node[] at (0, -1) { $p(\theta)$ };
  
\end{tikzpicture}

\end{document}  